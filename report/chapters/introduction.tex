\chapter{Introduction}

As urbanization continues to expand, integrating sustainable green spaces into densely populated areas has become a growing challenge. Rooftop solar panels, already serving as key assets in renewable energy production, present an opportunity to co\-function as natural green spaces, contributing to urban biodiversity and improved air quality. 
However, maintaining vegetation in these environments is an exhausting task, as overgrown plants can cast shadows on photovoltaic panels, reducing their efficiency but the human interference within the green spaces preferentially is kept to a minimum. 

To address this an autonomous robot capable of maintaining rooftop vegetation will be developed. The robot will be equipped with advanced decision-making capabilities to identify and trim specific plants without hindering the photovoltaic panels performance. To ensure optimal functionality and provide real-time monitoring, a digital twin will be developed to simulate, control, and oversee the robot's actions. This innovative solution combines robotics, artificial intelligence, and digital simulation to create a sustainable coexistence between green infrastructure and renewable energy technology. 

Controlling robots is complex and requires a lot of knowledge and experience, among the systems that reduce the complexity of controlling robots is the Robot Operating System (ROS2). 
Therefore, as a starting point, this project aims to learn and understand ROS2 and its applications in a concrete robotic application. Therefore, a simplified version of the envisioned autonomous robot will be used in this project.

\section{Project aim}

Within the ROS2 ecosystem, a digital twin will be developed, focusing on a control system managing the DT and the robot.
 ROS2 defines the components, interfaces and tools for building advanced robots. A robot is built out of three primary elements: actuators, sensors, and a control system. Actuators enable the robot to perform movements, sensors allow it to perceive its environment, and the control system functions as the robot's brain, processing inputs and coordinating actions.
ROS2 enables users to build these components and create a connection between them using ROS2 tools, which are called topics and messages. The visualization and simulation environment, such as rviz2 and gazebo, provides powerful tools for developing and testing robotic systems in a virtual space before deploying them in the real world \autocite{openroboticsROSHome}.



