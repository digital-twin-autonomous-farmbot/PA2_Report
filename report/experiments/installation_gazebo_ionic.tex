\section{Install Gazebo Ionic}

\subsection{Objective}
The inidial try to install gazebo was by using the defalt installation. Therefore the command:
\begin{lstlisting}[language=bash]
    sudo apt install ros-rolling-ros-gz 
\end{lstlisting}

was used. This installed a version von gazebo which was to old for the ubunt 24.04 operation system. 
Due to this the error
\begin{lstlisting}[language=bash]
    [GUI] [Err] [Ogre2RenderEngine.cc:1301]  Unable to create the rendering window: 
    OGRE EXCEPTION(3:RenderingAPIException): currentGLContext was specified with no current GL context 
    in GLXWindow::create at ./.obj-x86_64-linux-gnu/gz_ogre_next_vendor-prefix/src/gz_ogre_next_vendor/RenderSystems/GL3Plus/src/windowing/GLX/OgreGLXWindow.cpp (line 165)
\end{lstlisting}
    

accuret. This leat to the interution of the simulation, before the simulation could be loaded the programm was killed. 

\subsection{Procedure}
As describet on the official website of gazebo, a specific version of gazebo had to be installed. For ROS2 Rolling the Gazebo Ionic was compartible. 
After removing the humble version of gazebo, the ionic version was installed by following the instructions of the official website: 

\begin{lstlisting}
    https://gazebosim.org/docs/latest/install_ubuntu/
\end{lstlisting}
