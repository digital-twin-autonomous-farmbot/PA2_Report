\section{Install ROS2 on Ubuntu 24.04}

\subsection{Objective}
The Objective of this experiment was to install ROS2 rolling, on the controll unit (Laptop HP with Ubuntu 24.04) and the Raspberry Pi (also with Ubuntu 24.04). 
The main issue was to creat a connention between the machines. 

\subsection{Procedure}
The first step was to install ROS2 rolling desktop on both machines. For that the instructions from the ros2 website where used. 
\url{https://docs.ros.org/en/rolling/Installation/Ubuntu-Install-Debs.html}

After the installation the source command for ROS2 setup file was added to the .bashrc file on both devices. 
\begin{lstlisting}[language=bash]
    #oben file via editor
    nano ~/.bashrc
    # add the command on the botton of the file
    source /opt/ros/rolling/setup.bash
\end{lstlisting}

To creat a connention between the machines the following steps were follwed, on both machines.
The ROS\_IP and the ROS\_DOMAIN\_ID where addet after the source command for the ros2 setup file, within the .bashrc file.
The ROS IP adresse represented the IP adresse of the current machine, and the ROS DOMAIN ID was the same domain number for both machines. 


\begin{lstlisting}[language=bash] 
    export ROS_DOMAIN_ID=0
    export ROS_IP=192.168.x.xx
    \end{lstlisting}

Then the .bashrc file was sourced on both machines with the following command:
\begin{lstlisting}[language=bash] 
    source ~/.bashrc
    \end{lstlisting}


    \subsection{Testing}
To test if the two machines could communicate a demo node of ROS2 was used. 
Therefore the two commands:
\begin{lstlisting}
    ros2 run demo_nodes_cpp talker
    ros2 run demo_nodes_py listener
\end{lstlisting}

Where used, one machine was sending a messeage with the talker, and the other reseved the message with the listener. 

\subsection{Additional}
If the connection still can't be astablished the following steps can be tryed:

1. Check if your SSH-Connection works correctly. 

2. Check your firewall status 

3. When using a virtuell machine, make sure your controll unit is connetet through etahl cable to the network, and not over wlan. 
